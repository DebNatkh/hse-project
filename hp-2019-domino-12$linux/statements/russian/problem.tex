\begin{problem}{Домино}{стандартный ввод}{стандартный вывод}{1 секунда}{256 мегабайт}

Вася установил на телефон игру, где в каждой клетке полоски размером $2 \times N$ записано целое число. Цель игры состоит в том, чтобы накрыть часть клеток доминошками размерами $2 \times 1$ так, чтобы сумма чисел на не покрытых доминошками клетках была минимальной. 

Доминошки можно поворачивать горизонтально или вертикально, они не могут накладываться. Обязательно использовать все имеющиеся доминошки.

\InputFile
В первой строке вводится два числа $N$ и $K$ ($1 \leq N \leq 2\cdot 10^5$, $0 \leq K \leq 2\cdot 10^5$, $0 \leq N \times K \leq 2\cdot 10^5$, $N \geq K$)~--- размер полоски и количество имеющихся доминошек.

В следующих $N$ строках вводятся по $2$ целых числа, записанных на полоске. Числа не превосходят $10^9$ по модулю.

\OutputFile
Выведите $N$ строк по 2 числа в каждой~--- описание расположения доминошек на полоске. Каждая клетка должна описываться либо числом от 1 до $K$~--- номером доминошки, которой она накрыта, либо числом 0, в случае, если она не накрыта доминошкой.

Если ответов несколько~--- выведите любой из них.

\Examples

\begin{example}
\exmpfile{example.01}{example.01.a}%
\exmpfile{example.02}{example.02.a}%
\end{example}

\Note
Решения, верно работающие при $N \le 20$ будут получать не менее 50\% баллов.

\end{problem}

