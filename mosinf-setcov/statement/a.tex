\begin{problem}{Римское сложение}{Входные данные}{Результат}{}

На факультативе по истории математики Вася должен научиться суммировать римские числа. Для записи римского числа используются римские цифры: I~--- 1, V~--- 5, X~--- 10, L~--- 50, C~--- 100, D~--- 500, M~--- 1000.

Римские числа от 1 до 10 записывается по следующим правилам: если число от 1 до 3, то просто выписываем I количество раз, равное числу; если число равно 4, то оно записывается как IV (записанная слева от V I означает, что мы вычитаем 1 из 5); 5 записывается как V; числа от 6 до 8 включительно записываются как VI, VII и VIII; число 9 записывается как IX (десять без одного); число 10 записывается как X.

Аналогичным образом записывается значение разряда десятков, только вместо I, V и X используются X, L, и C соответсвенно. В разряде сотен используются цифры C, D и M. И, наконец, в разряде тысяч используются только буквы M, в количестве не более трех штук.

Для окончательной записи римского числа выписывается римское представление его разряда тысяч, затем разряда сотен, затем разряда десятков, затем разряда единиц. Подробнее о Римских числах можно прочитать на Википедии.

Васе необходимо просуммировать пары римских чисел и посчитать результат суммирования в десятичной системе. К сожалению, примеров очень много, а Вася не очень внимателен. Помогите ему решить эту задачу.

В первом тесте записано 10 пар римских чисел. Оценка за этот тест: 30 баллов. За каждое правильно вычисленное выражение начисляется 3 балла. Проверка осуществляется в режиме online (результат виден сразу).

Во втором тесте записано 700 пар римских чисел. Оценка за этот тест: 70 баллов. За каждое неправильно вычисленное выражение оценка снижается на 3 балла, однако не может стать меньше нуля. Во время тура проверяется, что сданный файл содержит 700 чисел. Проверка правильности ответа осуществляется в режиме offline (результат виден после окончания тура).

\Examples
\begin{example}
\exmp{I	VIII
MCMXCIX	XX
XV	V
CMXCIX	I
}{9
2019
20
1000
}%
\end{example}

\end{problem}
