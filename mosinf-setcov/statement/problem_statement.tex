

\begin{problem}{Покрытие}{Входные данные}{Результат}{}

Аудиторный университета вуза состоит из $n$ уникальных аудиторий. В один день в данном вузе планируется провести $m$ мероприятий. Для проведения $i$-е мероприятия потребуется ровно $k_i$ аудиторий с номерами $x_{i\, 1}, x_{i\, 2}, \ldots x_{i\, k_i}$. Разумеется, одна аудитория может быть занята не более чем под одно меропрудовиятие. Ваша задача ~--- выбрать какой-то поднабор мероприятий, которые будут проведены в данный день таким образом, чтобы каждая аудитория была занята не более чем одним мероприятием, а все требования по аудториям к проводимым мероприятиям были удовлетворены. Вам требуется максимизировать число занятых аудиторий.

\InputFile
Входной файл содержит на первой строке одно число $t$~--- число различных тестовых данных. Затем следует $t$ описаний тестов. Описание $i$-го теста в первой ствоей строке содержит два числа $n_i$ и $m_i$, описывающие число аудиторий и мероприятий соответственно. Далее следует $m_i$ строк, описывающие мероприятия. $j$-я строка описывает $j$-е мероприятие и начинается с числа $k_j$~--- количества требуемых аудиторий для данного мероприятия, затем через пробел следуют $k_j$ азличных чисел~--- номера аудиторий, которые требуются для проведения $j$-то мероприятия. Аудитории нумеруются с единицы. 

\OutputFile
Для каждого из $t$ тестов выведите ответ в следующем формате: в первой строке выведите $ans_i$ (от $0$ до $m_i$)~--- число мероприятий , которые можно провести в один день, а во второй строке $ans_i$ различных чисел от $1$ до $n$, разделённых пробелами~--- номера этих мероприятий

% \Scoring

\Examples
\begin{example}
\exmp{
2
3 3
2 1 2
2 2 3
2 3 1
4 3
3 1 2 3
2 1 2
2 3 4
}{
1
2
2
2 3
}%
\end{example}

{\noindent\bf\problemsectionfont\textsf{Система оценки}}

Оценка за каждый тест вычисляется по формуле $5 \times\left(\frac{\text {ParticipantSolution}}{\text {BestSolution}}\right)^{3}$, где $\text{ParticipantSolution}$~--- суммарное число занятых аудиторий в решении участника, а $\text{BestSolution}$~--- суммарное число занятых аудиторий в лучшем среди участников и жюри решении.

Оценка за группу тестов является суммой оценок по тестам данной группы.

В первой группе тестов $t = 6,\, m_i \leqslant 20,\, n_i \leqslant 100$. Максимальная оценка за эту группу: $30$ баллов. 

Во второй группе тестов $t = 14,\, m_i \leqslant 100,\, n_i \leqslant 10\,000$. Максимальная оценка за эту группу: $30$ баллов. Во время тура проверяется. что сданный файл оответсвует формату выходных данных чисел. Проверка правильности ответа осуществляется в режиме offline (результат виден после окончания тура).

Если ответ на хотя бы один из тестов группы не удовлетворяет описанному выше формату, решение получит 0 баллов. 

\end{problem} 
